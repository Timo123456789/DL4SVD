%!TEX root = ../thesis.tex
\chapter{Results}
\label{ch:results}

\begin{itemize}
    \item Gleitenden Mittelwert bei Darstellung mit Window Size 20 um das Grundrauschen zu minimieren
\end{itemize}
\section{Oriented Bounding Boxes and Axis Aligned Bounding Boxes}
\begin{itemize}
    \item obb weil performance bei Map40-95 in allen Epochen besser
    \item Map bei obb zwiscehn 0.5 und 0.6; bei abb bei 0.35 und 0.45
    \item Bei gleichen Bounding box koordinaten ist obb bei 0.55 und 0.59 und abb 0.59 und 0.6
    \item bessere Performance von abb wahrscheinlich weil geringe Änderungen in der Orientierung der Bounding Boxen eine hohe Auswirkung auf die Genauigkeit der Boxen haben
\end{itemize}

\section{RGBIR vs. ... (map5-95)}
\begin{itemize}
    \item Signifikant schlechtere map bei Modellen mit NDVI Kanal im Datensatz, unabhängig von Kanalanzahl (RGB NDVI und GB NDVI)
    \item RGB: Performance relativ ähnlich, alles wenig schwankungen. Zusätzlicher kanal hat wenig auswirkugnen
\end{itemize}
