%!TEX root = ../thesis.tex
\chapter{Fundamentals}
\label{ch:fundamentals}

\section{Computer Vision}
Computer Vision ist ein Teilbereich der künstlichen Intelligenz, der eng mit der Bildverarbeitung und dem maschinellen Lernen verknüpft ist. Dabei wird die Rohdatenerfassung durch Verfahren erweitert, die digitale Bildverarbeitung, Mustererkennung, maschinelles Lernen und Computergrafik kombinieren. Ziel ist es, Maschinen die menschliche Fähigkeit zu verleihen, aus Bildern Informationen zu extrahieren und zu interpretieren \cite{Wiley2018}. Zur Informationsgewinnung und zur Simulation menschlicher visueller Wahrnehmung werden Algorithmen sowie optische Sensoren eingesetzt \cite{Matiacevich2013}. Es lässt sich zwischen Bildbeschaffung und Bildanalyse unterscheiden. Wichtige Komponenten der Bildanalyse sind unter anderem:
\begin{itemize}
    \item \textbf{Bilderzeugung:} Das Speichern eines Objekts als digitales Bild.
    \item \textbf{Bildverarbeitung:} Verbesserung der Bildqualität zur Erhöhung des Informationsgehalts.
    \item \textbf{Bildsegmentierung:} Trennung des Objekts vom Hintergrund.
    \item \textbf{Bildvermessung:} Ermittlung signifikanter Bildpunkte (sogenannter Features).
    \item \textbf{Bildinterpretation:} Ableitung semantischer Informationen aus den Bilddaten \cite{Mery2013}. 
\end{itemize}



\section{Machine Learning}
\section{Deep Learning}
\subsection{Backpropagation}
\subsection{Vanishing}
\subsection{Exploring Gradients}
\subsection{Hyperparameter}
\subsection{Overfitting}







\section{YOLOv9}

\section{Evaluation Metrics}
\subsection{N Fold Cross Validation}
\subsection{Mean Average Precision}
\subsubsection{mAP@50}
\subsubsection{mAP@90-95}

% If you use underscores in section titles or anywhere in text, write \_ instead of _
\subsubsection{mAP@50}
\subsubsection{mAP@90-95}


