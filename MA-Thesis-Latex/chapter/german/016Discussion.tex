%!TEX root = ../thesis.tex
\chapter{Discussion}
\label{ch:discussion}
\begin{itemize}
    \item unterteilung analog zu  und Fig. \ref{fig:Flowchart} und Chap. \ref{ch:results}
\end{itemize}

\section*{Oriented Bounding Boxes and Axis Aligned Bounding Boxes}
\begin{itemize}
    \item in \ref{fig:comparison_bb_format} ist deutlich zu sehen, dass \acrshort{obb} deutlich exakter entlang der Objektgrenzen verlaufen und weniger umgebenes Areal umfassen; keine Überlappung ist wichtig, sorgt möglicherweise für eine bessere Erkennungsleistung bei nahe aneinander liegenden Objekten
    \item exaktere Objektgrenzen sieht man auch an geringeren Flächeninhalten über alle Objekte hinweg (s.\ref{fig:bbox_area})
    \item bessere Performance von \acrshort{abb} auf dem \acrshort{obb} Modell Wahrscheinlich dadurch, dass kleine Abweichunge/Änderungen in der Orientierung große Auswirkungen auf die \acrshort{mAP} haben
    \item \acrshortpl{obb} sind schmaler als \acrshort{abb} sodass dort eine höhere ungeauigekt beim \acrshort{mAP} herrscht
    \item \todo{Schreiben warum der eine Fold meistesn besser war als der andere? keine idee warum.. \ref{tab:best_folds_area} (vllt weil die Verteilung bei F 0 und 1 am besten für das Training geeignet ist?)}
\end{itemize}



\section*{Permutation Experiments}
\begin{itemize}
    \item das RGIR Modell hat beste MAP Leistung auf Testfold, B ist möglicherweise ein unwichtiger Kanal zur Fahrzeugdetektion
    \item NDVI könnte für Noise im Datensatz sorgen, sodass Fahrzeuge schlechter erkannt werden, deswegen performance beider modelle so schlecht
    \item IRGB ist mittelprächtig, weil möglicherweise der rote kanal für Fahrzeugdetektion wichitg ist 
    \item Fold 2 und 3 scheinen für den Testdatensatz die beste Verteilung der Trainingsdaten zu haben, da diese die höchsten MAP Werte haben, bei validierungsmodell auf Validierungsdatensatz hat jeder fold mindestens einmal am besten abgeschnitten (-> Verteilung spielt dort weniger eine große Rolle) \todo{Verteilungen genauer analysieren}
\end{itemize}
\begin{itemize}
    \item \todo{Verteilung genauer analysieren anhand performance bei Klassenerkennung bei Confusion Matrix (?Irrelevant weil immer auf Testdaten und dort ist die Objektanzahl immer gleich?)}
    \item absolute Zahlen statt nur Prozentzahlen analysieren; geringe Anzahl bei Klassen kann hohe Auswirkungen auf die Prozentzahlen haben (absolute Zahlen angucken?; dadurch Aussagekraft beurteilen)
\end{itemize}
\begin{itemize}
    \item \acrshort{RGBNDVI} zeigt schlechtere Performance bei Ship, Vehicle und Plane;background wird als van fehlklassifiziert; 
    \item 12\% Unterschied bei Plane kommt wohl exakt aus der fehlklassifizierung des plane als background
    \item beide NDVI Modelle performen schlechter als RGBIR; liegt möglicherweise an Noise durch NDVI die den Algorithmus beeinflusst
    \item 
\end{itemize}
\section*{Ablation Studies}
\begin{itemize}
    \item Ablation: erst Verbesserung njr bei 3 kanalälen erwähnen dann 3+1 auflisten
\end{itemize}
\section*{Implication}
\begin{itemize}
    \item alle Flugzeuge werden unabhängig von der Kanalpermuation und sogar bei einzelnen Kanälen erkannt, wahrscheinlich wegen der prägnanten Form 
    \item 
\end{itemize}

\todo{Rückgriff auf State of Research nciht vergessen? Fahrzeugverteilung in gesamter Region etc}
\todo{Räumlcihen Kontext beachten; Vorteile von Spatial Kontext einfügen}