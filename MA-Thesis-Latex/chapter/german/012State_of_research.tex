%!TEX root = ../thesis.tex
\chapter{State of research}
\label{ch:state_of_research}

Mit Luftbildern oder Satellitenbildern können bereits Massengräber detektiert werden \cite{Kalacska2006}. 
Dieser Ansatz kann in nahezu Echtzeit oder in der Vergangenheit angewendet werden \cite{Kalacska2006}. 
Er basiert auf Satellitenbildern, die aus verschiedenen Bändern bestehen. 
Man kann ihn auch zur Detektion von geheimen Massengräbern nutzen \cite{Kalacska2006}. 
Möglicherweise ist ein Umbau zur Detektierung von Massengräbern in der Ukraine möglich \cite{Kalacska2006}.

Weitere Ansätze umfassen die Erkennung von Einzelgräbern mithilfe von Air Borne Imagery \cite{Leblanc2014}.

Außerdem wurden Satellitenbilder von chinesischen Krematorien genutzt, um die offiziellen Daten zur Covid Mortalität der Regierung zu überprüfen. 
Dies konnte in Kombination mit Augenzeugenbefragungen dafür genutzt werden, die offiziellen Zahlen zur Sterblichkeit anzuzweifeln \cite{Spiegel_article}.

Mit Satellitenbildern konnten bereits Zerstörungen von Hüttenstrukturen in Darfur detektiert werden \cite{Knoth2017}. 




RGB Bilder von Drohnen können in den meisten Szenarien gut für Objekterkennung eingesetzt werden. 
Wärme (IR) kann die Möglichkeiten zur Objekterkennung in der Nacht oder bei verdeckten Objekten erweitern. 
Ein Problem ist der Mangel an verfügbaren Trainingsdaten für IR Bilder von Drohnen. 


