%!TEX root = ../thesis.tex
\chapter{Introduction}
\label{ch:intro}
\todo{sämtliche Abkürzungen mit glocaries einbinden und verwalten}
\todo{mAP Bezeichnugn eindeutig machen strukturiert machen überall!!}
\section{Motivation}
%Proposal Version
    % Durch die leichte Verfügbarkeit von hochauflösenden Satelliten- und Luftbildern ist Detektion von Objekten weltweit möglich. Da außerdem die hohe Rechenleistung, die für maschinelles Lernen erforderlich ist, immer preiswerter wird, wird die Anwendung von künstlicher Intelligenz zur vollautomatischen Analyse immer leichter. \\
    % Das Monitoring von zerstörten (militärischen) Gerät oder Gebäudestrukturen in Konfliktregionen, wie Darfur \cite{Knoth2017} oder der Ukraine, kann durch künstliche Intelligenz unterstützt werden. Hier können Luftbiler gut genutzt werden, weil das Erstellen von Fotos vor Ort durch die Sicherheitslage am Boden in Konfliktregionen gefährlich sein kann. \\
    % Das Monitoring von militärischem Gerät an Ländergrenzen könnte benutzt werden um eine genauere Lageeinschätzung zu generieren, ob ein Konflikt bevorsteht. Es ist auch möglich die Arbeit von menschlichen Analysten unterstützen, da die Modelle Vorschläge für mögliche Detektionen und Klassifizierungen von Objekten liefern. \\
    % Maschinelles Lernen kann zur Automatisierung angewendet werden, um größere Datensätze zu verarbeiten. Eine Herausforderung hier besteht im Training der Algorithmen, da die Verfügbarkeit von Trainingsdaten limitiert ist.\\

    % Ein Anwendungszenario wäre die Auswertung des menschlichen Mobilitätsverhaltens im Rahmen des Zählen von Autos auf großen öffentlichen Parkplätzen. Hier kann auch ein Tracking von Verkehrsaufkommen durch die KI landesweit erfolgen, wenn Fahrzeuge auf einem Satellitenbild detektiert werden können. Dieser Ansatz kann genutzt werden um abzuschätzen ob mehr Parkflächen erforderlich sind oder die Anzahl der aktuellen Flächen ausreicht. Weitere Fragestellungen wären, ob es möglich ist fahrende und stehende Autos zu unterscheiden. 

    % Außerdem kann in dieser Arbeit evaluiert werden, ob die Präzision der  Objekterkennung und -klassifzierung von Fahrzeugen in elektro-optischen, multi-spektralen Luftaufnahmen durch die Nutzung von mehr Bildkanälen (NIR, IR) verbessert werden kann. Dies kann durch einen Vergleich der Ergebnisse beim Trainieren desselben Deep Learning Modells geschehen, welches jeweils nur mit 3 und einmal mit  mehr Kanälen trainiert wurde. Da Fahrzeuge relativ klein auf hochauflösenden Satellitenbildern dargestellt werden, kann auch evaluiert werden, wie gut sich ein Deep Learning Modell zur Klassifizierung sehr kleiner Objekte eignet, bzw. ob ein weiterer Kanal die Präzision verbessert.
    % \todo{kann yolov9 kleine objeket auf satbildern zuverlässig erkennen? Vergleich mit Josis Thesis?}

%Ohne Josis Thesis


% Durch die zunehmende Verfügbarkeit hochauflösender Satelliten- und Luftbilddaten ist die weltweite Detektion von Objekten technisch möglich geworden. Gleichzeitig sinken die Kosten für die hohe Rechenleistung, die für maschinelles Lernen erforderlich ist, sodass der Einsatz von Methoden der Künstlichen Intelligenz (KI) zur vollautomatischen Analyse solcher Daten kontinuierlich erleichtert wird. Damit entstehen neue Möglichkeiten zur effizienten Auswertung großer Mengen an Bilddaten, die sowohl im zivilen als auch im militärischen Kontext von erheblicher Relevanz sind. \\

% Ein bedeutsamer Anwendungsbereich liegt im Monitoring von zerstörten militärischen Anlagen oder Gebäudestrukturen in Konfliktregionen, wie beispielsweise in Darfur \cite{Knoth2017} oder der Ukraine. Der Einsatz von KI-gestützter Bildanalyse kann hier wertvolle Unterstützung bieten, da eine direkte fotografische Dokumentation vor Ort aufgrund der unsicheren Sicherheitslage in Konfliktgebieten oftmals nicht möglich oder mit erheblichen Risiken verbunden ist. Luft- und Satellitenbilder eignen sich daher in besonderem Maße für die Informationsgewinnung. \\

% Darüber hinaus könnte die kontinuierliche Überwachung von militärischem Gerät an Ländergrenzen zur Verbesserung der Lageeinschätzung beitragen, indem potenzielle Eskalationen oder bevorstehende Konflikte frühzeitig erkannt werden. Ein entscheidender Vorteil der KI-gestützten Analyse besteht hierbei in der Möglichkeit, menschliche Analysten bei ihrer Arbeit zu entlasten, indem Modelle Vorschläge für Detektionen und Klassifizierungen liefern, die anschließend überprüft und validiert werden können. Auf diese Weise können bestehende Arbeitsabläufe effizienter gestaltet und Entscheidungsträger schneller mit relevanten Informationen versorgt werden. \\

% Neben diesen sicherheitsrelevanten Fragestellungen existieren auch vielfältige zivile Anwendungsbereiche. So kann die Analyse des Mobilitätsverhaltens der Bevölkerung durch die Erfassung von Fahrzeugen auf großen öffentlichen Parkplätzen unterstützt werden. Mithilfe automatisierter Detektions- und Klassifikationsmethoden lassen sich Fahrzeuge auf Satelliten- oder Luftbildern identifizieren und zählen, was wiederum Rückschlüsse auf die Auslastung von Parkflächen ermöglicht. Diese Informationen können beispielsweise genutzt werden, um den Bedarf an zusätzlichen Parkflächen zu bestimmen oder die Effizienz der bestehenden Infrastruktur zu evaluieren. Darüber hinaus eröffnet sich die Möglichkeit, das Verkehrsaufkommen auf nationaler Ebene zu quantifizieren, sofern eine kontinuierliche Fahrzeugdetektion auf Satellitenbildern implementiert wird. Eine weiterführende Fragestellung ist die Differenzierung zwischen fahrenden und stehenden Fahrzeugen, die zusätzliche Einblicke in Mobilitätsmuster und Verkehrsströme liefern könnte. \\

% Eine zentrale Herausforderung bei der Anwendung maschinellen Lernens auf hochauflösende Fernerkundungsbilder liegt jedoch in der begrenzten Verfügbarkeit geeigneter Trainingsdaten sowie in der Tatsache, dass Fahrzeuge auf Satellitenaufnahmen aufgrund des Maßstabs relativ klein dargestellt werden. Die präzise Erkennung und Klassifizierung dieser kleinen Objekte erfordert angepasste Deep-Learning-Methoden, die in der Lage sind, auch feine Strukturen zuverlässig zu erfassen. \\

% Im Rahmen dieser Arbeit soll daher untersucht werden, inwieweit die Präzision der Objekterkennung und -klassifizierung von Fahrzeugen in elektro-optischen und multispektralen Luft- und Satellitenaufnahmen verbessert werden kann, wenn zusätzliche Bildkanäle wie beispielsweise NIR- oder IR-Bänder einbezogen werden. Hierfür wird ein Deep-Learning-Modell sowohl mit konventionellen dreikanaligen (RGB) Daten als auch mit multispektralen Daten trainiert und die Ergebnisse miteinander verglichen. Auf diese Weise kann evaluiert werden, ob die Erweiterung der Spektralinformation die Erkennungsleistung für kleine Objekte wie Fahrzeuge verbessert. Gleichzeitig wird überprüft, wie gut sich moderne Deep-Learning-Ansätze grundsätzlich zur Detektion und Klassifizierung sehr kleiner Objekte eignen und welche Potenziale sich durch den Einsatz multispektraler Fernerkundungsdaten eröffnen.

%ChatGPT Vorschlag mit Josis Thesis
% Durch die zunehmende Verfügbarkeit hochauflösender Satelliten- und Luftbilddaten ist die weltweite Detektion von Objekten technisch möglich geworden. Gleichzeitig sinken die Kosten für die hohe Rechenleistung, die für maschinelles Lernen erforderlich ist, sodass der Einsatz von Methoden der Künstlichen Intelligenz (KI) zur vollautomatischen Analyse solcher Daten kontinuierlich erleichtert wird. Damit entstehen neue Möglichkeiten zur effizienten Auswertung großer Mengen an Bilddaten, die sowohl im zivilen als auch im militärischen Kontext von erheblicher Relevanz sind. \\

% Ein bedeutsamer Anwendungsbereich liegt im Monitoring von zerstörten militärischen Anlagen oder Gebäudestrukturen in Konfliktregionen, wie beispielsweise in Darfur \cite{Knoth2017} oder der Ukraine. Der Einsatz von KI-gestützter Bildanalyse kann hier wertvolle Unterstützung bieten, da eine direkte fotografische Dokumentation vor Ort aufgrund der unsicheren Sicherheitslage in Konfliktgebieten oftmals nicht möglich oder mit erheblichen Risiken verbunden ist. Luft- und Satellitenbilder eignen sich daher in besonderem Maße für die Informationsgewinnung. \\

% Darüber hinaus könnte die kontinuierliche Überwachung von militärischem Gerät an Ländergrenzen zur Verbesserung der Lageeinschätzung beitragen, indem potenzielle Eskalationen oder bevorstehende Konflikte frühzeitig erkannt werden. Ein entscheidender Vorteil der KI-gestützten Analyse besteht hierbei in der Möglichkeit, menschliche Analysten bei ihrer Arbeit zu entlasten, indem Modelle Vorschläge für Detektionen und Klassifizierungen liefern, die anschließend überprüft und validiert werden können. Auf diese Weise können bestehende Arbeitsabläufe effizienter gestaltet und Entscheidungsträger schneller mit relevanten Informationen versorgt werden. \\

% Neben diesen sicherheitsrelevanten Fragestellungen existieren auch vielfältige zivile Anwendungsbereiche. So kann die Analyse des Mobilitätsverhaltens der Bevölkerung durch die Erfassung von Fahrzeugen auf großen öffentlichen Parkplätzen unterstützt werden. Mithilfe automatisierter Detektions- und Klassifikationsmethoden lassen sich Fahrzeuge auf Satelliten- oder Luftbildern identifizieren und zählen, was wiederum Rückschlüsse auf die Auslastung von Parkflächen ermöglicht. Diese Informationen können beispielsweise genutzt werden, um den Bedarf an zusätzlichen Parkflächen zu bestimmen oder die Effizienz der bestehenden Infrastruktur zu evaluieren. Darüber hinaus eröffnet sich die Möglichkeit, das Verkehrsaufkommen auf nationaler Ebene zu quantifizieren, sofern eine kontinuierliche Fahrzeugdetektion auf Satellitenbildern implementiert wird. Eine weiterführende Fragestellung ist die Differenzierung zwischen fahrenden und stehenden Fahrzeugen, die zusätzliche Einblicke in Mobilitätsmuster und Verkehrsströme liefern könnte. \\

% Eine zentrale Herausforderung bei der Anwendung maschinellen Lernens auf hochauflösende Fernerkundungsbilder liegt jedoch in der begrenzten Verfügbarkeit geeigneter Trainingsdaten sowie in der Tatsache, dass Fahrzeuge auf Satellitenaufnahmen aufgrund des Maßstabs relativ klein dargestellt werden. Die präzise Erkennung und Klassifizierung dieser kleinen Objekte erfordert angepasste Deep-Learning-Methoden, die in der Lage sind, auch feine Strukturen zuverlässig zu erfassen. \\

% Vorarbeiten in diesem Themenfeld haben bereits gezeigt, dass Deep-Learning-Methoden wie YOLO grundsätzlich in der Lage sind, kleine Objekte in hochauflösenden Fernerkundungsbildern zu detektieren. Dabei konnte nachgewiesen werden, dass durch gezielte Anpassung von Hyperparametern wie Rastergröße, Trainingsdauer und Learning Rate die mittlere Average Precision (mAP) für sehr kleine Objekte verbessert werden kann. Insbesondere beim Airbus Ship Detection Datensatz wurden durch diese Optimierungen deutliche Leistungssteigerungen in den Klassen kleiner Objekte erzielt. \\

% Allerdings zeigten dieselben Untersuchungen auch die Grenzen dieses Ansatzes auf: Für komplexere Datensätze mit größeren Bildausschnitten und einer Vielzahl verschiedener Objektklassen, wie etwa den DOTA-Datensatz, ließ sich keine signifikante Verbesserung der Erkennungsleistung erzielen. Daraus ergibt sich die Schlussfolgerung, dass die Nutzung von YOLO für komplexe Szenarien mit mehreren Objektklassen und sehr kleinen Zielobjekten nicht zielführend ist. \\

%Mit Josis THesis

Durch die zunehmende Verfügbarkeit hochauflösender Satelliten- und Luftbilddaten ist die weltweite Detektion von Objekten technisch möglich geworden. Gleichzeitig sinken die Kosten für die erforderliche Rechenleistung, sodass der Einsatz von Methoden der Künstlichen Intelligenz (KI) zur vollautomatischen Analyse solcher Daten kontinuierlich erleichtert wird. Damit eröffnen sich neue Möglichkeiten zur effizienten Auswertung großer Mengen an Bilddaten, die sowohl im zivilen als auch im militärischen Kontext von erheblicher Relevanz sind. \\

Ein bedeutsamer Anwendungsbereich liegt im Monitoring von zerstörten militärischen Anlagen oder Gebäudestrukturen in Konfliktregionen, wie beispielsweise in Darfur \cite{Knoth2017} oder der Ukraine. Der Einsatz von KI-gestützter Bildanalyse kann hier wertvolle Unterstützung bieten, da eine direkte Dokumentation vor Ort aufgrund der unsicheren Sicherheitslage oftmals nicht möglich oder mit erheblichen Risiken verbunden ist. Luft- und Satellitenbilder eignen sich daher besonders gut zur Informationsgewinnung. \\

Darüber hinaus könnte die kontinuierliche Überwachung von militärischem Gerät an Ländergrenzen zur Verbesserung der Lageeinschätzung beitragen, indem potenzielle Eskalationen oder bevorstehende Konflikte frühzeitig erkannt werden. Ein entscheidender Vorteil der KI-gestützten Analyse besteht darin, menschliche Analysten zu entlasten, indem Modelle Vorschläge für Detektionen und Klassifizierungen liefern, die anschließend überprüft und validiert werden. Auf diese Weise können Arbeitsabläufe effizienter gestaltet und Entscheidungsträger schneller mit relevanten Informationen versorgt werden. \\

Auch im zivilen Bereich eröffnen sich vielfältige Anwendungsfelder. So kann die Analyse des Mobilitätsverhaltens der Bevölkerung durch die Erfassung von Fahrzeugen auf großen öffentlichen Parkplätzen unterstützt werden. Mithilfe automatisierter Detektions- und Klassifikationsmethoden lassen sich Fahrzeuge auf Satelliten- oder Luftbildern identifizieren und zählen, was Rückschlüsse auf die Auslastung von Parkflächen erlaubt. Diese Informationen können beispielsweise genutzt werden, um den Bedarf an zusätzlichen Parkflächen zu bestimmen oder die Effizienz der bestehenden Infrastruktur zu evaluieren. Zudem eröffnet sich die Möglichkeit, das Verkehrsaufkommen auf regionaler oder nationaler Ebene zu quantifizieren, sofern eine kontinuierliche Fahrzeugdetektion implementiert wird. Eine weiterführende Fragestellung ist die Differenzierung zwischen fahrenden und stehenden Fahrzeugen, die zusätzliche Einblicke in Mobilitätsmuster und Verkehrsströme liefern kann. \\

Eine zentrale Herausforderung bei der Anwendung maschinellen Lernens auf hochauflösende Fernerkundungsbilder liegt in der begrenzten Verfügbarkeit geeigneter Trainingsdaten sowie darin, dass Fahrzeuge aufgrund des Maßstabs relativ klein dargestellt werden. Die präzise Erkennung und Klassifizierung dieser kleinen Objekte erfordert angepasste Deep-Learning-Methoden, die auch feine Strukturen zuverlässig erfassen können. \\

Vorarbeiten in diesem Themenfeld, wie sie in einer vorherigen Bachelorarbeit durchgeführt wurden, zeigen, dass Deep-Learning-Methoden wie \acrfull{YOLO} (v3) kleine Objekte in einfachen Szenarien grundsätzlich erkennen können. Dabei konnte durch die Anpassung von Hyperparametern wie Rastergröße, Trainingsdauer und Learning Rate die \acrfull{mAP} für sehr kleine Objekte verbessert werden. Allerdings wird im Anwendungsfall der Schiffsdetektion deutlich, dass die erzielte mAP kritisch zu bewerten ist und die Eignung von YOLOv3 für eine zuverlässige Detektion von kleinen Objekten nicht uneingeschränkt gegeben ist \cite{Balzer2022}. \\

Die vorliegende Masterarbeit knüpft an diese Ergebnisse an und überträgt die Fragestellung auf den Bereich der Detektion und Klassifikation von kleinen Fahrzeugen in hochauflösenden multispektralen Luftbilddaten. Dabei wird eine neuere Version von \acrfull{YOLO} (v9) eingesetzt, die Verbesserungen in der Netzwerkarchitektur aufweist. Zusätzlich wird untersucht, inwieweit die Integration elektro-optische und multispektrale Luftbildaufnahmen  die Präzision bei der Detektion und Klassifikation sehr kleiner Objekte steigern kann. Auf diese Weise erweitert die Arbeit bestehende Forschung sowohl inhaltlich (Fokus auf Fahrzeuge) als auch methodisch (Verwendung moderner YOLO-Architektur und multispektraler Daten) und adressiert die Forschungslücke der Objekterkennung in komplexen Szenarien.



\section{Structure of the thesis}