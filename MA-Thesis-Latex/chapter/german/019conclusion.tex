%!TEX root = ../thesis.tex
\chapter{Conclusion and Outlook}
\label{ch:conclusion}

\section{Conclusion}
Die in dieser Arbeit formulierten Forschungsfragen konnten erfolgreich beantwortet werden. 
Es zeigte sich, dass zusätzliche Bildkanäle wie Infrarot (IR) die Detektionsleistung insbesondere bei kleinen Objekten verbessern können, während spektrale Indizes wie der NDVI nicht in jedem Fall vorteilhaft sind. 
Die Experimente verdeutlichen, dass bereits RGB-Daten eine robuste Grundlage für die Erkennung kleiner Fahrzeuge bieten, wenngleich IR-Daten die Genauigkeit weiter erhöhen. 
Zudem wurde ersichtlich, dass die Verfügbarkeit umfangreicher, qualitativ hochwertiger Trainingsdaten einen größeren Einfluss auf die Modellleistung hat als die Erweiterung der Eingangskanäle. 
Die Untersuchungen bestätigen, dass \textsc{YOLOv9} für die Detektion kleiner Objekte geeignet ist und architekturelle Weiterentwicklungen gegenüber älteren Versionen wie \textsc{YOLOv3} zu messbaren Verbesserungen in Klassifikation und Genauigkeit führen.

\section{Outlook}
Zukünftige Arbeiten sollten eine systematische Integration weiterer spektraler Indizes und zusätzlicher Kanäle evaluieren, um deren Einfluss auf die Modellleistung präziser zu bestimmen. 
Darüber hinaus bietet die Anwendung und Modifikation weiterer \textsc{YOLO}-Varianten sowie der Vergleich mit zweistufigen Detektoren (z.\,B. \textsc{R-CNN}-basierte Verfahren) ein vielversprechendes Untersuchungsfeld. 
Ein weiterer Ansatz besteht in der Analyse von Skalierungseffekten, etwa durch die Reduktion der räumlichen Auflösung auf Satellitenniveau, um die Übertragbarkeit auf Rohdaten realer Systeme (z.\,B. Airbus, 30\,cm/Pixel) zu evaluieren. 
Auch die Rolle objektspezifischer Merkmale, wie etwa die Fahrzeugfarbe, sollte vertieft untersucht werden. 
Langfristig eröffnet insbesondere die Nutzung multitemporaler Daten neue Perspektiven, da sich damit dynamische Muster und Veränderungen im Raum-Zeit-Kontext (z.\,B. Pendlerverkehr versus Dauerabstellung) erfassen lassen.

% \section{Conclusion}
% \begin{itemize}
%     \item Forschungfragen konnten beantwortet werden. 
%     \item Zusätzliche Bildkänale konnen Ergebnisse verbessern (IR) aber auch verschlechtern (NDVI)
%     \item Leistung für die Erkennung kleiner Objekte ist auch mit reinen RGB Daten gut, auch wenn IR die Genauigkeit nochmals verbessert
%     \item Erhöhte Menge an Hochwertigen Trainingsdaten schlägt die hinzunahme eines weiteren Kanals im Sinne der "Verbesserung der Modellperformance"
%     \item Grundsätzlich eignet sich YOLOv9 für die Erkennung kleiner Objekte, Auswirkungen der Veränderungne an der Architektur auf Klassifizierung und Genauigkeit kleiner Objekte ist sichtbar (im vgl. zu YOLOv3)
% \end{itemize}
% \section{Outlook}
% \begin{itemize}
%     \item mehr indizes (wie NDVI ) nutzen ujdn vergleichen
%     \item noch mehr kanäle und infos in yolo reinwerfen
%     \item andere yolo versionen verwenden
%     \item yolo selbst modifizierne
%     \item 2 Stage Detectoren ( RCNN und so) ausprobieren und vergleichen
%     \item luftbilder runtersaklieren (von 12.5cm/pixel auf 30 cm/pixel (wie bei Airbus) um zu schauen wie die Leistung von YOLO da aussieht -> Eignung für Satellitenrohdaten kann evaluiert werden)
%     \item weitere Untersuhcungen ob Fahrzeugfarbe einfluss auf Erkennungsleistung DL Modell aht
%     \item  \textbf{Zeitlicher Kontext (Future Work):} 
%     Multitemporale Bilder ermöglichen es, dynamische Muster zu erkennen 
%     (z.\,B. regelmäßige Pendlerfahrzeuge vs. dauerhaft abgestellte Fahrzeuge). 
%     Damit können Veränderungen im Raum-Zeit-Kontext erfasst werden
% \end{itemize}

