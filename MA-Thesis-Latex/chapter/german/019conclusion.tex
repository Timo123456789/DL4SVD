%!TEX root = ../thesis.tex
\chapter{Conclusion and Outlook}
\label{ch:conclusion}

\section{Conclusion}
\begin{itemize}
    \item Forschungfragen konnten beantwortet werden. 
    \item Zusätzliche Bildkänale konnen Ergebnisse verbessern (IR) aber auch verschlechtern (NDVI)
    \item Leistung für die Erkennung kleiner Objekte ist auch mit reinen RGB Daten gut, auch wenn IR die Genauigkeit nochmals verbessert
    \item Erhöhte Menge an Hochwertigen Trainingsdaten schlägt die hinzunahme eines weiteren Kanals im Sinne der "Verbesserung der Modellperformance"
    \item Grundsätzlich eignet sich YOLOv9 für die Erkennung kleiner Objekte (vlg zu yolov3?)
\end{itemize}
\section{Outlook}
\begin{itemize}
    \item mehr indizes (wie NDVI ) nutzen ujdn vergleichen
    \item noch mehr kanäle und infos in yolo reinwerfen
    \item andere yolo versionen verwenden
    \item yolo selbst modifizierne
    \item 2 Stage Detectoren ( RCNN und so) ausprobieren und vergleichen
    \item luftbilder runtersaklieren (von 12.5cm/pixel auf 30 cm/pixel (wie bei Airbus) um zu schauen wie die Leistung von YOLO da aussieht -> Eignung für Satellitenrohdaten kann evaluiert werden)
    \item weitere Untersuhcungen ob Fahrzeugfarbe einfluss auf Erkennungsleistung DL Modell aht
    \item  \textbf{Zeitlicher Kontext (Future Work):} 
    Multitemporale Bilder ermöglichen es, dynamische Muster zu erkennen 
    (z.\,B. regelmäßige Pendlerfahrzeuge vs. dauerhaft abgestellte Fahrzeuge). 
    Damit können Veränderungen im Raum-Zeit-Kontext erfasst werden
\end{itemize}

