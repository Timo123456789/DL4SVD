%!TEX root = ../thesis.tex
\chapter{Fazit und Ausblick}
\label{ch:conclusion}
\section{Fazit}
{ 
    Zusammenfassend kann man sagen, dass das Formähnlichkeitsmaß nach der Vereinfachung von DCE gering genug ist, um Objekttracking zu ermöglichen. Beim Einsatz der größeren Modelle steigt das Formähnlichkeitsmaß an, dies ist aber zu vernachlässigen, da die Klassifizierung der Objekte genauer erfolgt.  Damit sind die in der Einleitung gesteckten Ziele dieser Arbeit erreicht worden. \\
	Im Vergleich zu bereits existierender Forschung konnte gezeigt werden, dass die DCE für Objekttracking nicht nur bei intelligenten Verkehrsleitssytemen genutzt werden kann und damit einen Mehrwert bietet. Formbasiertes Objekttracking ist mit der DCE im Bereich Verkehrstracking möglich, weil die Formähnlichkeit zwischen den detektierten Objekten über den Verlauf eines Videos beibehalten wird. Ein Nebeneffekt der DCE sind die vereinfachten Umrisse der detektierten Verkehrsteilnehmer, die dadurch anonymisiert werden. \\
\\
	

    }
\section{Ausblick}
{
	Aktuell sind feste Punktgrenzen für die einzelnen Klassen zur Berechnung mit der DCE festgelegt. Dies kann durch einen Wert ersetzt werden, der die Ähnlichkeit des vereinfachten Polygons zum Ursprungspolygon misst. Dadurch wird eine bedarfsbezogene Vereinfachung des Polygons für jede Klasse ermöglicht. Dies wird von \citeauthor{Latecki2003} in \citetitle{Latecki2003} \citep{Latecki2003} näher erläutert. \\
	Außerdem wäre eine weitere Möglichkeit, in den Code weitere Klassen mit individuellen Punktgrenzen einzufügen. Dies könnte für eine größere Abdeckung mehrerer Szenarien genutzt werden. Begrenzt wird dieses Vorhaben lediglich durch die Anzahl der 80 Klassen, die YOLO unterscheiden kann. \\
	Da die Prozessierungsgeschwindigkeit von Python begrenzt ist, kann eine effizientere Implementierung der DCE in C oder C++ erfolgen, um ein echtzeitfähiges System zu ermöglichen. \\
	Dieses echtzeitfähige System könnte in einer Art \glqq Verifying Tracker\grqq{} eingesetzt werden, bei dem gleichzeitig mehrere Kamerasignale aus verschiedenen Blickwinkeln verarbeitet werden. Es kann ebenfalls eine Signalkomprimierung stattfinden, um die Verbindung zwischen den einzelnen Systemen zu stabilisieren.  Außerdem kann durch die verschiedenen Kamerablickwinkel berechnet werden, ob das betrachtete Objekt der Klasse entspricht, die getrackt werden soll. Dadurch ließe sich eine höhere Sicherheit bei der Objektdetektion ermöglichen. 

}

