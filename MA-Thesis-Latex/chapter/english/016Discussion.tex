%!TEX root = ../thesis.tex
\chapter{Discussion}
\label{ch:discussion}


The experiments comparing the use of axis-aligned bounding boxes (ABB) with \acrfull{obb} show interesting results. OBBs run more precisely along the object boundaries and capture less surrounding area, which in theory should lead to greater accuracy in object detection. This is reflected in the smaller surface areas of the bounding boxes. Surprisingly, however, the ABB model shows better performance in terms of mean average precision (mAP). One possible explanation is that small deviations in the orientation of the OBB can greatly reduce the overlap with the ground truth. Even minimal angular deviations lead to significantly lower intersection-over-union values, which reduces the mAP of the OBB models. ABB thus offers more stable performance, while OBB adapts the bounding boxes more accurately to the object and can therefore use computing resources more efficiently.
 
Analysis of the channel permutations shows that the choice of spectral channels has a significant impact on object detection performance. Models that use the IR channel in combination with RGB achieve the highest mAP (0.59–0.63), while the blue channel appears to contribute little to vehicle detection. This may be because the blue spectral range (420–490 nm) provides only limited information compared to red (650–780 nm), green (490–575 nm) and infrared (>800 nm) \cite{bfs_Strahlung}. The NDVI channel, on the other hand, reduces the performance of the models, presumably due to additional noise in the dataset. Vehicles are easily recognisable even without NDVI, as distinguishing them from vegetation does not offer any decisive advantage for detection and instead leads to confusion. %\todo{refer to example images?}

The investigation of the primary permutations shows that the IRGB model performs the worst. Models such as RGBIR benefit from the combination of true colour channels and IR, while NDVI reduces detection performance. All models show difficulties in separating background and objects, especially for the classes \textit{Ship} and \textit{Van}. Aircraft, on the other hand, are reliably detected, presumably due to their characteristic shape and unique environment, such as airports or water surfaces.
 
The ablation studies confirm that individual channels have only a limited influence on detection performance. Only the combination of several channels leads to a significant improvement in mAP. Single channels achieve values below 0.55, while combinations such as RGIR achieve values between 0.59 and 0.63. Minimal fluctuations in the quantiles between the individual channels indicate that even a single channel (except NDVI) is fundamentally suitable for object recognition. The NDVI channel delivers the worst performance at 0.42–0.44, which cannot be explained by the radiometric resolution (8 bits). Vehicles of the class \textit{Car} are robustly recognised, even without channel combination, presumably due to the high amount of training data and their characteristic shape. The greatest differences occur with background objects, with IR delivering the best results and combinations of channels enabling further improvements. Difficulties arise in particular with generic classes such as \textit{Vehicle} and with confusion between \textit{Ship} and \textit{Camping Car}, as the shape or partial coverage of the objects makes differentiation difficult. Distinguishing between individual vehicle types other than \textit{Car} remains a challenge due to limited data.

Against the backdrop of these results, it can be concluded that the use of \acrshort{DL}-based methods to generate suggestions for object detection and classification in vehicles has the potential to reduce the workload of human analysts. At the same time, the results show that the human factor continues to be of central importance. Compared to human analysts, the models tend to make incorrect classifications or fail to reliably identify relevant objects in a considerable number of cases. This highlights both the limitations of current model quality and the need for a hybrid approach in which algorithmic suggestions are reviewed and validated by human expertise.

In addition to these model-intrinsic limitations, scene-specific factors also have a significant impact on detection performance. Vehicles in structured suburban scenes are relatively easy to recognise, while vehicles in scrap yards or overgrown areas are difficult to detect. Colour also plays a role: white vehicles on grey asphalt are easily recognised, while grey or black vehicles are more difficult to distinguish from the background. This correlates with the distribution of vehicle colours in the USA, where white (24\%) and black (23\%) are particularly common \cite{abc_utah}.

YOLO models are suitable for vehicle detection and, to a limited extent, for more complex multi-class scenarios, provided that the resolution and number of spectral channels are sufficient. Furthermore, vehicle types must be clearly distinguishable in shape, colour and size to avoid confusion. With high-resolution data (12.5 cm × 12.5 cm per pixel), reliable detection of different vehicle classes is possible, especially for cars, which achieve an mAP of over 80\%. This is probably due to the large amount of training data. \acrshort{obb} correspond to the definition of \textit{very small objects} according to \citeauthor{Chen2017} \cite{Chen2017} and thus enable the detection of small objects with \acrshort{YOLO}v9. \todo{more perhaps} \acrshort{YOLO}v9 can reliably locate and categorise \Acrlong{GT} objects, which allows applications such as the analysis of vehicle distributions across entire regions, especially in combination with regular satellite overflights \cite{planet_labs, airbus_neo}. Such analyses of satellite images of Chinese crematoria were used, for example, during the COVID-19 pandemic to verify the government's official COVID mortality data by evaluating the utilisation of parking spaces (for hearses). In combination with eyewitness interviews, this could be used to cast doubt on the official mortality figures \cite{Spiegel_article}. \todo{Update link to article} 

%\todo{more perhaps? EVIDENCE for the matter}
The use of multispectral channels and spatial context information offers several advantages. The model can recognise typical patterns, such as car parks or port areas, and provide semantic information from multispectral channels, which makes the classification more robust against camouflage, shadows or colour variations. Channels such as \acrshort{NDVI} or \acrshort{IR} also enable the assessment of object condition and use. The combination of multiple channels improves generalisation to different environments such as urban, forest or desert, thus increasing robustness to new test areas. Beyond pure vehicle detection, categories such as vehicle type (based on shape and size), condition (colour, e.g. rust detection) or usage context (area classification, car in densely populated urban environment is more likely than aircraft) can be distinguished. Combining this with other data sources, such as \acrshort{SAR}, \acrshort{LiDAR} or elevation models, enables applications such as traffic density analysis, infrastructure condition or usage patterns.



